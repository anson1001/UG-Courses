\documentclass[10pt,landscape]{article}
\usepackage{amssymb,amsmath,amsthm,amsfonts}
\usepackage{multicol,multirow}
\usepackage{calc}
\usepackage{ifthen}
\usepackage[landscape]{geometry}
\usepackage[colorlinks=true,citecolor=blue,linkcolor=blue]{hyperref}

%new packages
\usepackage{calligra}
\DeclareMathAlphabet{\mathcalligra}{T1}{calligra}{m}{n}
\DeclareFontShape{T1}{calligra}{m}{n}{<->s*[2.2]callig15}{}
\newcommand{\scripty}[1]{\ensuremath{\mathcalligra{#1}}}



\ifthenelse{\lengthtest { \paperwidth = 11in}}
    { \geometry{top=.5in,left=.5in,right=.5in,bottom=.5in} }
	{\ifthenelse{ \lengthtest{ \paperwidth = 297mm}}
		{\geometry{top=1cm,left=1cm,right=1cm,bottom=1cm} }
		{\geometry{top=1cm,left=1cm,right=1cm,bottom=1cm} }
	}
\pagestyle{empty}
\makeatletter
\renewcommand{\section}{\@startsection{section}{1}{0mm}%
                                {-1ex plus -.5ex minus -.2ex}%
                                {0.5ex plus .2ex}%x
                                {\normalfont\large\bfseries}}
\renewcommand{\subsection}{\@startsection{subsection}{2}{0mm}%
                                {-1explus -.5ex minus -.2ex}%
                                {0.5ex plus .2ex}%
                                {\normalfont\normalsize\bfseries}}
\renewcommand{\subsubsection}{\@startsection{subsubsection}{3}{0mm}%
                                {-1ex plus -.5ex minus -.2ex}%
                                {1ex plus .2ex}%
                                {\normalfont\small\bfseries}}
\makeatother
\setcounter{secnumdepth}{0}
\setlength{\parindent}{0pt}
\setlength{\parskip}{0pt plus 0.5ex}
% -----------------------------------------------------------------------

\title{PHYS3053}

\begin{document}

\raggedright
\footnotesize

\begin{center}
     \Large{\textbf{Chapter 5}} \\
\end{center}
\begin{multicols}{3}
\setlength{\premulticols}{1pt}
\setlength{\postmulticols}{1pt}
\setlength{\multicolsep}{1pt}
\setlength{\columnsep}{2pt}

\section{5.1 The Lorentz Force Law}
Empirically, for a moving charge in a magnetic field, we have 
$$F_{mag} = Q(v \times B)$$
The generalized magnetic force is
$$F_{mag} = \int I\times B dl = I \int dl \times B$$
The magnetic force does \emph{NO} work.

\subsection{Currents}
Line Currents: $I = v\lambda$\\
Surface Currents Density: $K = \sigma v$, $K = \frac{dI}{dl_\perp}$\\
(for solenoids of n winds per length, $K=nI \vec{\phi}$)\\
Volume Current Density: $J = \rho v,$ $J = \frac{dI}{da_\perp}$\\
\subsection{Conservation of Charge}
\begin{align*}
    \int_{S(V)} J\cdot da &= \frac{d}{dt}\int_V \rho d\tau\\
    \nabla \cdot J &= \frac{\partial \rho}{\partial t}
\end{align*}

\section{5.2 Biot-Savart Law}
For $\frac{\partial J}{\partial t} = 0$, we have,
$$B = \frac{\mu_0}{4 \pi} I \int \frac{dl' \times \hat{\scripty{r}}}{\scripty{r}^2} = \frac{\mu_0}{4\pi} \int \frac{J(r')\times B}{\scripty{r}^2}d\tau'$$\\
Field of a straight wire: $B = \frac{\mu_0 I}{2\pi d}$\\
Field of a circular wire: $B = \frac{\mu_0 I}{2} \frac{R^2}{(R^2+z^2)^{3/2}}$


\section{5.3 The Divergence and Curl of B}
\subsection{The Divergence}
$$\nabla \cdot B = 0$$\\
\subsection{The Curl: Ampere's Law}
\begin{align*}
    \nabla \times B &= \mu_0 \vec{J}\\
    \oint B \cdot dl &= \mu_0 I_{enc}
\end{align*}
Ampere's law is particularly useful for
\begin{enumerate}
\begin{multicols}{2}
    \item Infinite straight wires
    \item Infinite planes
    \columnbreak
    \item Infinite solenoids
    \item Toroids
\end{multicols}
\end{enumerate}
Field of a solenoid: $B = \begin{cases}
    \mu_0nI \hat{z} &\quad\text{inside}\\
    0 &\quad\text{outside}
\end{cases}$\\
Field of a toroid: $B = \begin{cases}
    0 &\quad\text{inside the coil}\\
    \frac{\mu_0 N I}{2 \pi s} &\quad\text{outside the coil}
\end{cases}$

\raggedcolumns\columnbreak

\section{5.4 Magnetic Vector Potential}
$$B = \nabla \times A$$
\begin{align*}
    \nabla \times B = \nabla \times ( \nabla \times A) = \nabla (\nabla \cdot A) - \nabla^2A &= \mu_0 \vec{J}\\
    \text{(3 Poisson's equations corres. x,y,z) \quad} \nabla^2 A &= \mu_0 \vec{J}
\end{align*}
$$A(r) = \frac{\mu_0}{4\pi}\int\frac{J}{\scripty{r}^2}d\tau' = \frac{\mu_0}{4\pi}\int\frac{K}{\scripty{r}^2}da' = \frac{\mu_0}{4\pi}\int\frac{I}{\scripty{r}^2}dl'$$
By using $B = \nabla \times A$, Ampere's law can also be applied on $A$,
$$\oint A \cdot dl = \Phi = \int B\cdot da$$
Typically, the direction of A mimics that of currents.\\
\subsection{Boundary Conditions}
\begin{align*}
B_{above} - B_{below} &= \mu_0 ( \mathbf{K} \times \hat{n} )\\
A_{above} - A_{below} &= 0 \text{\quad (continuous potential)}\\
\frac{\partial A_{above}}{\partial n} - \frac{\partial A_{below}}{\partial n} &= - \mu_0 K
\end{align*}

\subsection{Multipole Expansion on Vector Potential}
$$\frac{1}{\scripty{r}} = \frac{1}{\sqrt{r^2 + r'^2 - 2rr'cos\alpha}} = \frac{1}{r} \sum_{n=0}^{\infty} (\frac{r'}{r})^n P_n(cos\alpha)$$
$$ A(r) = \frac{\mu_0 I}{4\pi}  \sum_{n=0}^{\infty} \frac{1}{r^{n+1}} \oint r'^n P_n(cos\alpha) dl'$$
where the first magnetic monopole term always vanishes as $ \oint dl' = 0$.
The dipole term is 
\begin{align*}
A_{dip}(r) &= \frac{\mu_0}{4\pi} \frac{\mathbf{m} \times \hat{r}}{r^2}\\
B_{dip}(r) &= \frac{\mu_0}{4\pi} \frac{3(\mathbf{m} \cdot r) \hat{r} - \mathbf{m}}{r^3}
\end{align*}
where $\mathbf{m}$ is the magnetic moment
$$ \mathbf{m} = I \int da = Ia $$

\raggedcolumns\columnbreak
.
\end{multicols}

\pagebreak

\begin{center}
     \Large{\textbf{Chapter 6}} \\
\end{center}
\begin{multicols}{3}
\setlength{\premulticols}{1pt}
\setlength{\postmulticols}{1pt}
\setlength{\multicolsep}{1pt}
\setlength{\columnsep}{2pt}



\end{multicols}

\end{document}