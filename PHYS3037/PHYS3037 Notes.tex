%% Hello! First of all, thanks for being interested in this little thing i
%% have made during my time learning LaTeX. Feel free to use it as you
%% want. I tried to comment everything, and I reduced the document to 
%% the minimum for it to work. Read the theorems.tex to learn more :)
%% If you want to change the color, see theorems.tex as well hehe

\documentclass[a4paper,12pt]{article}

% Minimum packages required for it to work (chosen by trial and error xd)
\usepackage{amssymb}
\usepackage{graphicx} 
\usepackage[dvipsnames]{xcolor}
\usepackage[most]{tcolorbox} 
\usepackage{fancyhdr} 
\usepackage{hyperref} 
\usepackage{tikz}

% Head of every page. Depending on the photo, u need to change both \put(x,y) and width=X mm

%\lhead{\begin{picture}(0,0) \put(0,-10){\includegraphics[width=20mm]{HeadPhoto.png}} \end{picture}}
%\rhead{\vspace{-3mm} -Anson- \vspace{3mm}}
\setlength{\headheight}{15pt}
\renewcommand{\headrulewidth}{0,5pt}
\pagestyle{fancy}



%% Possible commands

%   Definition:     \defin
%   Theorem:        \theo
%   Example:        \ex
%   Demonstration:  \demo


%  If u want to create a reference for that command,
%  write an 'r' at the end

%%  Exemples of use: 

% \defin, \theo and \ex have the same structure

% \defin{Name or title of the definition}{
%     A definition
% }

% \definr{Name or title of the definition}{Reference code}{
%     A definition
% }

% To reference, u can write \ref{label:Reference code}

% \demo has no title 

% \demo{
%   A demo.
% }




% The color of all the environments can be changed with this variable.
% To do that, just search the code of the color in hexadecimal and 
% change it at the end.

\definecolor{doccol}{HTML}{ff2b2b}

% Some nice colors I have tried :)
%   Light Blue:   32dfd8
%   Green:        32c12f
%   Violet:       c233c4
%   Red:          ff2b2b

% If you want to modify only one thing, try to change the values of
% colback, colframe, colbacktitle...



% ============================
% Definition
% ============================
\newtcbtheorem[auto counter]{mydefinition}{Definition}{
    enhanced,
    beforeafter skip = 0.6cm,
    colback=doccol!10,
    colframe= doccol!70!black!,
    colbacktitle = doccol!40,
    halign title = center,
    titlerule=0.5mm,
    toptitle = 0.8mm,
    titlerule style=doccol!70!black,
    coltitle = black,
    fonttitle=\bfseries,
    separator sign= \ -,
    }{defin}

\NewDocumentCommand{\defin}{m+m}{
    \begin{mydefinition}{#1}{}
        #2
    \end{mydefinition}
}

\NewDocumentCommand{\definr}{mm+m}{
    \begin{mydefinition}{#1}{#2}
        #3
    \end{mydefinition}
}


% ============================
% Theorem
% ============================

 \newtcbtheorem[auto counter]{mytheorem}{Theorem}{
 enhanced,
 beforeafter skip = 0.6cm,
 colback=white,
 colframe= doccol!70!black!,
 fonttitle=\bfseries,
 coltitle = black,
 attach boxed title to top center = {
    yshift=-3mm,
    yshifttext=-2mm
    },
 boxed title style = {
    colback=doccol!20!, 
    colframe = doccol!70!black!,
    },
 separator sign= \ -,
 % Here, you can change the position of the image. Depending of which one
 % you use, maybe you need to change the scale or rotation, i doubt 
 % the position needs to be changed
 overlay={
    \node at ([xshift=\tcboxedtitlewidth/2 + 0.1cm ,
    yshift = \tcboxedtitleheight/2 ]frame.north) 
    {\includegraphics[scale = 0.035]{Figures/umaru.png}};
    }
}{theo}

\NewDocumentCommand{\theo}{m+m}{
    \begin{mytheorem}{#1}{}
        #2
    \end{mytheorem}
}

\NewDocumentCommand{\theor}{mm+m}{
    \begin{mytheorem}{#1}{#2}
        #3
    \end{mytheorem}
}

% ============================
% Example
% ============================

\newtcbtheorem[auto counter]{myexample}{Example}{
    enhanced,
    beforeafter skip = 0.6cm,
    detach title, before upper={\tcbtitle\textbf{.}\quad},
    colback=doccol!10,
    frame hidden,
    after skip=10pt,
    borderline west={1mm}{0pt}{doccol!70!black!},
    coltitle = black,
    fonttitle=\bfseries,
    separator sign= \ -,
    }{ex}

\NewDocumentCommand{\ex}{m+m}{
    \begin{myexample}{#1}{}
        #2
    \end{myexample}  
}

\NewDocumentCommand{\exr}{mm+m}{
    \begin{myexample}{#1}{#2}
        #3
    \end{myexample}  
}



% ============================
% Demonstration
% ============================

\newenvironment{dem}{
    \vspace{-2pt}
    {\noindent{\it \textbf{Demonstration.}}}
    \tcolorbox[
        blanker,
        breakable,
        left = 5mm,
        parbox = false,
        before upper = {\parindent15pt},
        after skip = 15pt,
	borderline west={1mm}{0pt}{doccol!80!black!}]
}{
    \textcolor{doccol!40!white}{\hbox{}\nobreak\hfill$\blacksquare$} 
    \endtcolorbox
}

\NewDocumentCommand{\demo}{m}{
    \begin{dem}
        #1
    \end{dem}
}







 % Theorems styles and colors

\begin{document}

% Title
\begin{titlepage}
  \begin{center}

    
\textbf{\huge    Honors Quantum Mechanics I\\ PHYS3037}

    

\vspace{4cm}
Anson\\
2026 Spring    
  \end{center}
\end{titlepage}
\pagebreak

\tableofcontents
\pagebreak

\section{The Wave Function}
\subsection{The Schr\"odinger Equation}
In classical mechanics, we are interested in the position and momentum of a particle at any given time, with its time evolution goverened by Newton's laws.
In quantum mechanics, we obtain the \textbf{wave function} $\psi(x,t)$ which gives these informations, and its time evolution is governed by the \textbf{Schr\"odinger equation}.
\definr{Time-Dependent Schr\"odinger Equation}{TDSE}{
    The \textbf{time-dependent Schr\"odinger equation} (TDSE) is given by
    \[
        i \hbar \frac{\partial}{\partial t} \Psi(x,t) = -\frac{\hbar^2}{2m} \frac{\partial^2}{\partial x^2} \Psi(x,t) + V(x,t) \Psi(x,t)
    \]
    where $\Psi(x,t)$ is the wave function, $V(x,t)$ is the potential energy, $m$ is the mass of the particle, and $\hbar = 1.054573 \times 10^{-34}$ is the reduced Planck's constant.
}
\defin{Wave Function}{
    The \textbf{wave function} $\Psi(x,t)$ is a complex-valued function that contains all the information about a quantum system.
}
\subsection{The Statistical Interpretation}
\defin{Born's Statistical Interpretation}{
  Born's statistical interpretation of the wave function states that the probability density of finding a particle at position $x$ at time $t$ is given by $|\Psi(x,t)|^2$.
}
In case I do measure the position of the particle, the wave function will collapse to a delta function at the measured position, and the probability of finding the particle at that position is 1. After the measurement, the wave function will evolve according to the Schr\"odinger equation again, and the wave function will spread out again, and the probability of finding the particle at that position will be less than 1.
\subsection{Probability}
Here we define some useful concepts related to probability in quantum mechanics.
\begin{itemize}
    \item \textbf{Probability Density}: The probability density of finding a particle at position $x$ is given by $\rho(x)$.
    \item \textbf{Expectation Value}: The expectation value of a function $f(x)$ is given by $$\langle f(x) \rangle = \int_{-\infty}^{\infty} f(x) \rho(x) dx$$
    Note that the expectation value is the average of measurements on an ensemble of identically prepared systems, not the average of repeated measurements on one and the same system.
    \item \textbf{Variance}: The variance of a function $f(x)$ is given by $$\sigma^2 = \langle f(x)^2 \rangle - \langle f(x) \rangle^2$$
\end{itemize}
\subsection{Normalization}
\defin{Normalization}{
    A wave function $\Psi(x,t)$ is said to be \textbf{normalized} if it satisfies the condition
    \[
        \int_{-\infty}^{\infty} |\Psi(x,t)|^2 dx = 1
    \]
}
Real particles can only be squared-integrable, which means that the integral of the probability density over all space must not be 0 or diverge.
Thus, for we are only interested in real particles, we can obtain its normalized wave function. Moreover, it can be derived that the normalization condition is preserved by the Schr\"odinger equation.
\subsection{Momentum}
Now we know the expectation value of the position:
$$\langle x \rangle = \int_{-\infty}^{\infty} x |\Psi(x,t)|^2 dx$$
Let's use a naive approach and assumme $\langle p \rangle = m \frac{d}{dt} \langle x \rangle$. We can derive that
\begin{align*}
    \langle p \rangle &= m \frac{d}{dt} \langle x \rangle \\
    &= m \frac{d}{dt} \int_{-\infty}^{\infty} x |\Psi(x,t)|^2 dx \\
    &= m \int_{-\infty}^{\infty} x \frac{\partial}{\partial t} |\Psi(x,t)|^2 dx \\
    &= m \int_{-\infty}^{\infty} x \left( \Psi^* \frac{\partial \Psi}{\partial t} + \Psi \frac{\partial \Psi^*}{\partial t} \right) dx\\
    &= m \int_{-\infty}^{\infty} x \left( \Psi^* \frac{1}{i \hbar} \left( -\frac{\hbar^2}{2m} \frac{\partial^2 \Psi}{\partial x^2} + V \Psi \right) + \Psi \frac{-1}{i \hbar} \left( -\frac{\hbar^2}{2m} \frac{\partial^2 \Psi^*}{\partial x^2} + V \Psi^* \right) \right) dx\\
    &= \frac{\hbar}{2i} \int_{-\infty}^{\infty} x \left( \Psi^* \frac{\partial^2 \Psi}{\partial x^2} - \Psi \frac{\partial^2 \Psi^*}{\partial x^2} \right) dx\\
    &= \frac{\hbar}{2i} \int_{-\infty}^{\infty} x \frac{\partial}{\partial x}\left( \Psi^* \frac{\partial \Psi}{\partial x} - \Psi \frac{\partial \Psi^*}{\partial x} \right) dx\\
    &= \frac{\hbar}{2i} \left[ \left. x \left( \Psi^* \frac{\partial \Psi}{\partial x} - \Psi \frac{\partial \Psi^*}{\partial x} \right) \right|_{-\infty}^{\infty} - \int_{-\infty}^{\infty} \left( \Psi^* \frac{\partial \Psi}{\partial x} - \Psi \frac{\partial \Psi^*}{\partial x} \right) dx \right] \\
    &= \int_{- \infty}^{\infty} \Psi^* \left( -i\hbar \frac{\partial}{\partial x} \right) \Psi dx
\end{align*}
\textbf{Note} Here we use the conjugation of \nameref{defin:TDSE}, adding a minus sign to the time derivative of the wave function, and we also used integration by parts in the last step.

We call the operator $-i\hbar \frac{\partial}{\partial x}$ the \textbf{momentum operator}, the momentum operator in three dimensions is $\hat{p} = -i\hbar \nabla$.

To find the operator for classically defined variables, such as kinetic energy and angular momentum, we just use the formulas $T = \frac{p^2}{2m}$ and $\bf{L} = \bf{r} \times \bf{p}$, and substitute $p$ for $\hat{p} = -i \hbar \nabla$, getting
\begin{align*}
  \hat{T} &= -\frac{\hbar^2}{2m} \nabla^2\\
  \hat{\bf{L}} &= -i \hbar \bf{r} \times \nabla
\end{align*}
\subsection{The Uncertainty Principle}
Heisenberg's uncertainty principle states that the product of the uncertainties in position and momentum is greater than or equal to $\frac{\hbar}{2}$, which will be derived later.
For now, just have an intuitive understanding of it: the more precisely we know the position of a particle, the less precisely we can know its momentum, and vice versa. This is not a limitation of our measurement devices, but a fundamental property of nature.
For a wave, you precisely know its wavelength, but you have no idea where the wave is. For a particle, you precisely know its position, but you have no idea what its wavelength is. 

\pagebreak

\section{The Time-Independent Schr\"odinger Equation}



\end{document}