\documentclass[10pt,landscape]{article}
\usepackage{amssymb,amsmath,amsthm,amsfonts}
\usepackage{multicol,multirow}
\usepackage{calc}
\usepackage{ifthen}
\usepackage[landscape]{geometry}
\usepackage[colorlinks=true,citecolor=blue,linkcolor=blue]{hyperref}

%new packages
\usepackage{calligra}
\DeclareMathAlphabet{\mathcalligra}{T1}{calligra}{m}{n}
\DeclareFontShape{T1}{calligra}{m}{n}{<->s*[2.2]callig15}{}
\newcommand{\scripty}[1]{\ensuremath{\mathcalligra{#1}}}
\usepackage{tikz}
\usepackage{tikz-3dplot}
\usepackage{mathrsfs}
\usepackage{float}
\usepackage{wrapfig}



\ifthenelse{\lengthtest { \paperwidth = 11in}}
    { \geometry{top=.5in,left=.5in,right=.5in,bottom=.5in} }
	{\ifthenelse{ \lengthtest{ \paperwidth = 297mm}}
		{\geometry{top=1cm,left=1cm,right=1cm,bottom=1cm} }
		{\geometry{top=1cm,left=1cm,right=1cm,bottom=1cm} }
	}
\pagestyle{empty}
\makeatletter
\renewcommand{\section}{\@startsection{section}{1}{0mm}%
                                {-1ex plus -.5ex minus -.2ex}%
                                {0.5ex plus .2ex}%x
                                {\normalfont\large\bfseries}}
\renewcommand{\subsection}{\@startsection{subsection}{2}{0mm}%
                                {-1explus -.5ex minus -.2ex}%
                                {0.5ex plus .2ex}%
                                {\normalfont\normalsize\bfseries}}
\renewcommand{\subsubsection}{\@startsection{subsubsection}{3}{0mm}%
                                {-1ex plus -.5ex minus -.2ex}%
                                {1ex plus .2ex}%
                                {\normalfont\small\bfseries}}
\makeatother
\setcounter{secnumdepth}{0}
\setlength{\parindent}{0pt}
\setlength{\parskip}{0pt plus 0.5ex}
% -----------------------------------------------------------------------

\title{Modern Physics Cheat Sheet}

\begin{document}

\raggedright
\footnotesize

\begin{center}
     \Large{\textbf{Modern Physics Cheat Sheet}} \\
\end{center}
\begin{multicols}{3}
\setlength{\premulticols}{1pt}
\setlength{\postmulticols}{1pt}
\setlength{\multicolsep}{1pt}
\setlength{\columnsep}{2pt}

\section{10.1 Molecular Bonding and Spectra}
Potential of a molecule is\\ 
$$V = \frac{A}{r^n}-\frac{B}{r^m}$$
where $n>m$ and
$$F = -\frac{dV}{dr} = \frac{mB}{r^{m+1}} - \frac{nA}{r^{n+1}}$$
The \textbf{binding energy} is the energy to seperate two atoms.
The \textbf{equilibrium radius} is when $F=0$.

\begin{figure}[H]
\frame{\includegraphics[width=150pt]{04e4d3174014c2973f3184776f7554e}}
\centering
\end{figure}

\subsection{Rotational States}
By $E_{\text{rot}}=\frac{L^2}{2I}$ and $L=\sqrt{l(l+1)}\hbar$, we have quantized \textbf{rotational energy}
$$ E_{\text{rot}} = \frac{\hbar^2 l (l+1)}{2I}$$
take note that
$$ I = \int \rho r^2 dV = mr^2 $$

\subsection{Vibrational States}
Vibrational kinetic energy is by QM Oscillator,
$$E_\text{vibr} = (n+\frac{1}{2})\hbar \omega$$
$$\omega = \sqrt{\kappa/\mu}$$
by $\kappa = \frac{dF}{dr}$,
$$\omega = \sqrt{\frac{e^2}{2\pi\epsilon_0\mu r^3}}$$

\subsection{Vibration and Rotation Combined}
Consider the energy of photon emitted due to excitation.
$$E_\text{ph} = E_\text{rot} + E_\text{vibr}$$
Recall selection rules $\Delta n = \mathbb{Z}, \Delta l = \pm 1$. Gives spectral pattern

\begin{figure}[H]
\frame{\includegraphics[width=150pt]{f1068003fdbdd00485a565927166a3f}}
\centering
\end{figure}

The missing central peak has
$$f = \frac{1}{2\pi} \sqrt{\frac{\kappa}{\mu}}$$

a\\
\raggedcolumns\newcolumn
b\\
\raggedcolumns\newcolumn
c
\end{multicols}
\end{document}